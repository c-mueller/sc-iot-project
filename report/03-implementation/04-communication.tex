\subsection{Communication}\label{subsec:communication}

As mentioned in \cref{subsec:non-functional-requirements}, the communication between our components is ensured via a loosely coupled messaging protocol.
For this purpose, we use the Mosquitto\footnote{\url{https://mosquitto.org/}} message broker to send and receive JSON encoded data using the MQTT protocol.
In the following we explain the communication in more
detail.
MQTT requires subscribing to so-called topics to receive messages instead of using a unique recipient address such as an IP address.
The topics are designed to be extended for larger use cases.
Their names start with an identifier for the location e.g.~\texttt{room001} followed by \texttt{input} or \texttt{output} to specify the direction of the topic, based on the view of the core application.
In practice sensor data is always sent on an \texttt{input} topic and actuator messages are always sent on an \texttt{output} topic.
Finally, the last part of the topic name defines the
device or sensor the topic is assigned to, for example a topic for a temperature sensor ends with \texttt{/temperature}.
The exception for this topic syntax are sensors placed outside of the building, here the direction is not defined, since we do not have any actuators outside.
Some examples for valid topic names are:
\begin{itemize}
    \item \texttt{room001/input/temperature} is the topic assigned to the temperature sensor of the room with the identifier \texttt{room001}.
    \item The outside humidity sensor publishes data on the topic \texttt{outdoors/humidity}.
    \item \texttt{room001/output/air-conditioning} is used to address the air conditioning unit of \texttt{room001}.
\end{itemize}

The messaging endpoint component of the core application subscribes to all available sensor topics to receive all measurements from sensors.
Each sensor publishes and every actuator subscribes to its associated topic as described above.
On one hand, this is necessary to ensure that the core component can receive the data from all loosely coupled sensors.
On the other hand, this way all actuators can receive their instructions from the core application without talking to it directly.

The following listing illustrates a measurement message in JSON from the $CO_2$ sensor in \texttt{room001}.
It contains the location of the sensor, the type of sensor, the time at which the measurement was made and the measured value as a floating point number.

\begin{verbatim}
{
    "location": "room001",
    "timestamp": "2021-07-05T07:30:01Z",
    "sensortype": "CO2",
    "value": 143
}
\end{verbatim}
Looking at the other direction, from the core application to the actuators, we just send a message regarding the actuation state of the respective device.
When designing the initial idea of the system we thought of electrical relays or smart plugs as actuators that turn the device on or off depending on the messages from the core application.
We therefore did not have any further configuration attributes like ventilation speed or target temperature in mind.
Depending on the real world system adjustments in the core application may be necessary if the system should reconfigure these attributes too.

