\subsubsection{Communication}\label{subsubsec:communication}

As mentioned in chapter (TODO), the communication between our components
is ensured via the lightweight MQTT protocol.
For this purpose, we use the Mosquitto (TODO: link https://mosquitto.org/) message broker to send and receive our json (TODO: link https://www.json.org/json-en.html)encoded data.
In the following we will explain the communication in more
detail.
MQTT requires subscribing to so-called topics to receive messages instead of using a unique recipient address such as an IP address.
Our topics are designed to be extended for larger use cases.
The topics start with an identifier for the location e.g.~``room001'' followed by ``input'' or ``output'' to specifiy the content of the payload.
With that information we separate the payload into a message from a sensor or a message that is sent to an actuator.
The finest granularity is defined by the last part of the topic which defines the
name of a communication component such as ``temperature'' for a
temperature sensor.
An example for a valid topic is
``room001/input/temperature''.
Another example for an actuator topic is
``room001/output/air\_conditioning''.
There exist some special cases for
example when having outside sensors that do not have a roomID. In this specific case, the location will be ``outside'', followed by the sensorID in the next granularity.
The messaging endpoint component subscribes to all available topics to receive all measurements from sensors.
Each created sensor or actuator subscribes to the corresponding topic as described before.
This is necessary to ensure that the core component can reach all loosely coupled components.

The communication in general can be divided into different areas:
\begin{itemize}
    \item Sensor to MQTT broker
    \item MQTT broker to Actuator
    \item MQTT broker and Core
    \item Core to User Interface
\end{itemize}


The protocol for a sending sensor to the MQTT broker and therefore to
the Core is defined as followed.
A location attribute, timestamp
according to ISO 8601 (https://www.iso.org/standard/70907.html), the
sensortype and the sensor measurement referred to as ``value''.
An
example for a humditiy sensor within room001 can be seen below:

\begin{verbatim}
{
    "location": "room001",
    "timestamp": "2021-07-05T07:30:01Z",
    "sensortype": "humidity",
    "value": 12
}
\end{verbatim}

In the following we will describe the protocol for the core sending a
state change to an actuator.
The MQTTEndpoint looks up the corresponding
topic to an actuator and send a message with the state e.g.~true or
false and an optional targetTemperature attribute in case we have an
actuator with input integer value e.g.~when having air conditoning.
We
drop the attribute for ventilation or air-purifier actuators.
Since the
identification is already achieved by looking up the topics and sending
a message to a specific topic, an identifier of the actuator is not
necessary in the payload.
Am example for an actuator message can be seen below:

\begin{verbatim}
{
    "active": "true"
}
\end{verbatim}
