\subsection{Core}\label{subsec:core}

The application core is written using the C\# programming language and consists of three different parts.
Firstly The AI-planning which will be further elaborated in \cref{subsec:ai-planning}.
Additionally, a messaging endpoint component which has the job to receive incoming messages from the MQTT broker as well as send outgoing messages to this MQTT broker.
This communication will be explained in detail in \cref{subsec:communication}.
Furthermore, the application core has an HTTP API endpoint from which the User Interface, which is topic of \cref{subsec:user-interface}, can get user relevant information about the sensor measurements and current state of the systems actuators.

The communication inside the Core differs for the communication between the messaging endpoint and the AI-planning and the communication between the AI-planning and the HTTP API.
The AI-planning component and HTTP API do not communicate directly with each other.
Instead the AI-planning component saves some of the information about its current state, namely sensor data and actuator states, inside an external application state store.
In our implementation this is realized using a Redis cache, but this could also be substituted with a regular relational database such as any SQL database, given an implementation has been provided based on our given interface.
The HTTP API can then look up the saved information it needs and send it to the requester.
The communication between the messaging endpoint and the AI-planning is more tightly coupled in contrast.
These two components communicate directly through application intern function calls.
On one hand the AI-planning does this when it has found changes in the current actuator state after it has finished one run of its planning routine.
On the other hand the messaging endpoint does this periodically, so the AI-planning routine is not executed every time a new sensor measurement is sent to the MQTT broker.
