\subsection{Simulation}\label{subsec:simulation}

Of course, it was not possible for us to build the system in person
using actual senors and actuators.
Therefore, we had to consider virtual
i.e.~simulated actuators and sensors instead.
For this, we faced two
initial options:
\begin{itemize}
    \item Using an existing IoT simulation framework or platform
    \item Writing a simulation build for our purpose from the ground up
\end{itemize}

First we investigated potential simulation frameworks and platforms such
as IoTIFY (TODO) or the Microsoft Azure IoT Device Simulation (TODO), however we could
not identify a suitable solution that was not deprecated (Microsoft
Azure IoT Device Simulation) or had a decent free version.
We could not
even identify a platform that had acceptable cost associated with it,
however we did not investigate this far, since no one in our team wanted
to spend money on this.
The only remaining option was to build a
simulation ourselves suitable for our use case from the ground up

When implementing, we had several goals/requirements in mind:

\begin{itemize}
    \item
    Make the simulated devices as independent as possible while still
    allowing simple, mostly centralized, control of everything.
    \item
    Make the simulator reconfigurable.
    For example, it should be easy to
    add a new sensor, change the topic of a device or adjust the default
    value of a sensor.
    \item
    Make the simulator easy to deploy on either an x86-machine, a
    Raspberry Pi-like Single board computer or even the cloud
\end{itemize}

To fulfil the third goal, we could either use Docker or a programming
language that allows developers to produce assets that do not have any
external dependencies apart from their runtime, like the Java Runtime
Environment.
Our simulator was implemented using the Go programming
language, since it produces a single statically linked binary with none
or almost no external dependencies, depending on the compiler options
chosen.
Past personal experience has also shown that Go can be used to
implement applications such as simple HTTP APIs or CLI utilities
relatively quickly.
To realize the simulation, the Gin web-framework
(TODO https://github.com/gin-gonic/gin) and the Paho MQTT Go Client
(TODO https://github.com/eclipse/paho.mqtt.golang) were used to implement the
functionality.

To achieve the second goal, a YAML based configuration file was
constructed that allows the configuration of the simulator in the
expected manner.
The following listing shows a simplified version of the
configuration file for our simulator, in this sample the simulator only
simulates sensors for indoor and outdoor temperature and an actuator for
the ventilation.
The meaning of the values in the configuration are
explained as comments in the listing.

\begin{lstlisting}[language=yaml]
# The HTTP port to listen on
http_port: 8080
# The host name/IP of the MQTT broker
mqtt_endpoint: 127.0.0.1
# The host name/IP of the MQTT broker
mqtt_port: 1883
# The duration between simulated sensor measurements
sensor_update_interval: 1s
# The list of sensors to simulate
sensors:
  # The type of sensor supported values are:
  # Temperature, Humidity, CO2 and ParticulateMatter
- type: Temperature
  # The location identifier used by the simulator
  # internally, either Indoors or Outdoors
  place: Indoors
  # the unique name of the sensor, used for API access
  name: temperature-indoors
  # The initial sensor value to publish to the MQTT broker
  initial_value: 20
  # The name of the topic to publish on
  topic: room001/input/temperature
  # Model specific location identifier, this is not used by
  # the simulator itself it just passes it
  # through into the MQTT messages
  location: room001
- type: Temperature
  place: Outdoors
  name: temperature-outdoors
  initial_value: 20
  topic: outdoors/temperature
  location: outdoors
actuators:
  # The type of actuator, possible values:
  # Ventilation, AirConditioner, Heater, AirPurifier
- type: Ventilation
  # Same as Sensor
  name: ventilation
  # Same as Sensor
  topic: room001/output/ventilation
  # Same as Sensor
  location: room001
\end{lstlisting}

Once the program is launched, the config file is loaded and the
simulated sensors and actuators are initialized according to the config
file.

Every simulated device is kept independent of each other, meaning that
they all have their own connection to the MQTT broker as well as their
own thread/co-routine for publishing and subscribing the only reason for
keeping them within one application was the simplicity of deployment and
the ability to easily modify or access the device values and states from
one central location without the need of distributed communication, like
MQTT, gRPC or HTTP.

A sensor first publishes the value that has been set in the config file
until the user changes the sensor value using the UI or the API. The
current value is published within a time interval that is defined in the
configuration file.

The actuators work in a similar manner, but their value is not modified
by the simulator UI, instead the value has to be set using MQTT messages
from the topic the simulated actuator subscribes to.

Following this approach, the first goal has also been fulfilled.

\subsubsection{Simulator User Interface}\label{subsubsec:ui}

Of course, the values for the virtual sensors have to be set somehow. We
decided to implement this using a simple Web UI based on Angular. This
UI shows all sensors and allows the user to set the new value that will
be published by the simulator for the value of the specific sensor. The
Web UI also shows the current state of the actuators, i.e.~whether they
are active or not.

The Web UI communicates with the simulator backend using an HTTP based
REST-like API. The current value of an actuator or sensor is retrieved
by polling an HTTP endpoint.
We decided against the use of the more
efficient alternative of a Publish-Subscribe approach using web-sockets,
because we did not see any benefit, while the use of web-sockets or long
polling may reduce the load on the backend if many users are connected
to the UI, we only expect one or two concurrent users making high server
load caused by redundant requests no concern to us.
Doing it this way
also made the implementation of both the UI and the backend API less
complicated, which was way more important to us.

Once the simulation UI is compiled it only consists of static files,
that could be served using any web server such as Apache httpd, assuming
the path / URL to the API of the backend is set properly.
In order to
keep the deployment of the simulator as a whole simple, these static
files are directly served using the simulator backend.
The compiled
JavaScript binaries of the Web UI are also included in the binary of the
backend application.
To achieve this, a Go library called packr was
used.
It loads all files in a previously defined directory into a go
source file that defines the content of the directory as variables.
packr then exposes these variables using an interface that is part of
the Go standard library.
Once the interface has been instantiated, it
can be served through HTTP, making the directory or in our case the UI
accessible over HTTP.

When a Go application is compiled it usually consists of a single file
which is relatively large that includes everything, i.e.~the UI, the
Backend Code and the dependencies.
This is comparable to a Fat JAR in Java.


\subsubsection{Going into the real
world}\label{going-into-the-real-world}

The system in the current shape is pretty much usable in the real world,
assuming of course, real sensors and actuators are used.
In a real world
system one also has to reimplement the scripts for actuators and
sensors, since the simulated ones are not usable.
Also the system will
become more distributed in the real world, since one device handling all
6 Sensors and 4 actuators is unlikely.
The collection of data will
therefore be handled in multiple locations.
