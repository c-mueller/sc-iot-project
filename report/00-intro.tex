\section{System Introduction}\label{sec:system-introduction}

Increasing the well-being of people is an important topic of research.
Especially for office spaces, when companies want to increase the efficiency of their employees.
In general, with good air in terms of proper temperature, humidity and air purity, people feel more comfortable making them more productive.
It can also be considered healthier, since previous research has shown that high levels of particulate matter in urban locations, where many offices are located, can be considered carcinogenic~\cite{pehnec2020carcinogenic}.

Another application could be retail stores where customers may stay longer, return at a later date or even recommend the store to their friends and family if they feel more comfortable, which may lead to higher revenue for the retail business.

The system we present in this document is designed to be operated in a singe office space.
it can control the temperature, particulate matter and carbon dioxide ($CO_2$) levels in the room using a set of different sensors and actuators that allow the control of these parameters, like an air conditioner, and a ventilation system, which could also just be a window that can be opened and closed automatically.
Our system was implemented to control a single room, since we wanted to build a functional basis that can be scaled up in the future, to be used for multiple rooms or even whole buildings.

Based upon this rough problem outline we performed a system analysis, defining functional and non-functional requirements in \cref{sec:system-analysis}.
After the definition of the requirements we designed a system architecture, described in \cref{sec:system-architecture-design}, with the theoretical basis defined we started implementing the system itself, as well as a simulator to test the system without having a real world room to test in, as discussed in \cref{sec:system-implementation}.
Finally, we conclude the report with potential expansion opportunities in \cref{sec:discussion-and-conclusions}.

