\section{System Introduction}
Increasing the well-being of people is an important topic of research. Especially for office spaces, when companies want to increase the efficiency of their employees. In general, with good air in terms of proper temperature, humidity and air purity, people feel more comfortable and can be considered healthier and more productive than with poor quality air. \\
Another application could be retail stores where customers stay longer if they feel more comfortable, which could lead to higher revenue for the retail business. Of course, businesses also face great competition from online retailers and other stores where air monitoring and customer comfort could be the difference or advantage. \\
Since we are participating in the lecture "Smart Cities and IoT", we want to gain practical experience in addition to the theoretical basics taught in the lecture. In our project, we will consider a single office space that controls heating, air conditioning, ventilation and air purifier according to several sensor measurements, thus monitoring the air quality. Since this example can be easily scaled up to large enterprises with many office rooms, we can easily test the system using a single room as a POC. Moreover, we only need to change some thresholds to extend the scope to retail stores or other buildings. Based on this basic problem definition and the main scope of our work, in the following we will specify our use cases and requirements for a system that should automate such tasks as well as some additional tasks for a user.


% TRASH -----------------------------------------------------------------------------------------------------
% Increasing people's efficiency and comfort is an important topic of research.
% Making everyday life more comfortable and easier is a predominant aspect of development related to smart buildings and especially smart homes.
% Therefore, it is not surprising that companies offering solutions for smart cities and smart home devices have significant market growth.
%
% Since we are participating in the lecture "Smart Cities and IoT" we want to gain practical experience in addition to the theoretical fundamentals taught by the lecture.
% Since Smart Cities are just a bigger model or a further development of a smart home system, we will consider a smart home system as our project.
%
% Let's assume we live in an appartement with e.g. a kitchen, a bathroom, a living room and a bedroom.
% Since we have a huge music affinity we want to listen the whole day to music while moving in our appartement.
% Without any smart home functionality we would have to turn on the speakers in the room we are currently in or turn it off once again when leaving.
% Additionally, of course we would have to connect our smartphone to some speaker first and play our favorite music.
% Once we got established the connection and the music is playing we would have to regulate the volume of the music according to many factors such as noise in the room.
%
% In the following based on this basic problem definition and main scope of our work we want to specify our use cases and requirements for a smart home system which aims to automate such tasks as well as some additional ones for a user.
