\section{Discussion and Conclusions}\label{sec:discussion-and-conclusions}

As we have seen, we developed a system that can control the temperature, $CO_2$ level and particulate matter level in a room.
Improving the overall air quality using several quality improving actuators, like an air conditioner to reduce the temperature in a room, a heater to increase the temperature in a room, an air purifier to adjust the particulate matter level and the ventilation to bring the indoor levels to the outdoor levels.

To conclude, we have shown that an automatic system monitoring and adjusting air quality inside a room is possible and has high expendability for further additions.
We could see some of the functional and non-functional requirements of such a system.
All mandatory requirements as well as some more optional ones were implemented for this system by us.
Our architecture was also based on common and proven IoT and smart city design principles.
We have also shown how the implementation of such a system could be realized and how the AI-planning of such a system could look like as well as how this AI-planning can then be integrated into such a system.
Additionally, a way to simulate sensors and actuators without the need of physical ones was elaborated.
Furthermore, a way to integrate all different system components and for them to communicate with each other has been described.

As shown in this work, our current system should be able to operate in the real world, with some limitations, since the system in its current state can only turn actuators on or off.
This limitation of course has impacts on the usability of the system, since the actuated devices can only operate at a preset speed / level.
This could impact the comfort of people in the room if the devices are operating on a high level causing unwanted noises like loud fan noises, for example.
If the levels are set too low, the system may not operate at its full potential, which means it will take longer for the system to achieve the expected goal, potentially resulting in higher energy use.

\pagebreak

Of course our system can be expanded with many more devices.
Here are some example additions or improvements that could be made to the AAQC system:

\begin{itemize}
    \item One could also monitor the indoor humidity levels and could then use a humidifier and a dehumidifier to keep the humidity level comfortable.

    \item The parameters for actuators could be expanded, to set the ventilation speed depending on the severity of the difference between indoors and outdoor parameters, for example.

    \item The system could also detect human absence to save energy if no one is using the room.
    This can save energy because an unused room could have different, less strict levels in comparison to the time when a human is present.
    For example, the temperature may go down to 15 \textdegree C and up to 30 \textdegree C, instead of 20 \textdegree C and 22 \textdegree C, if no one is in the room for a longer time.

    \item The time and date could also be taken into account, as an example the upper and lower thresholds may be changed based on the time or even the day of the week.
    This could also work depending on the time of the year, or maybe in combination with a weather forecast to know which times of the day would be best or worst for ventilation.

    \item There could also be an override for ventilating the room before anyone enters, e.g. early in the morning, to lower the $CO_2$ level to a minimum.
    As a result, the change of an $CO_2$ emergency where we have to ventilate could be lowered significantly, since this is unwanted on very hot or cold days.

    \item The system could be expanded, so that it may be able to control multiple rooms or even a whole building.
    However, when doing so, the usage of a local or self maintained AI-planner is mandatory.

    \item The system could also use a different AI-planner instead, for example one could use the \texttt{ff}~(fast-forward)~planner\footnote{\url{https://fai.cs.uni-saarland.de/hoffmann/ff.html}} locally instead of sending everything to the online planner we used.
    Doing so would also make the system completely independent of the internet.
\end{itemize}

The above list is by no means completed, there are many further additions and refinements one could think of adding to this system.
